\documentclass{beamer}
\usepackage[utf8]{inputenc}
\usepackage{textpos}
\usepackage{minted}
\usepackage{etoolbox}
\AtBeginEnvironment{minted}{\fontsize{10}{10}\selectfont}

\usecolortheme{dove}

\title{An introduction to Git and version control}
\author{Matthias Nilsson}
\institute{Chalmers University of Technology}
\date{May 29, 2015}

\logo{\includegraphics[height=1cm]{sysbio.jpg}\vspace{230pt}}

\definecolor{chalmersblue}{RGB}{0,0,102}
\setbeamercolor{footlinecolor}{fg=white,bg=chalmersblue}

\makeatother
\setbeamertemplate{footline}
{
  \leavevmode%
  \hbox{%
    \begin{beamercolorbox}[wd=\paperwidth,ht=8ex,dp=1ex,right]{footlinecolor}%
      \includegraphics[height=0.7cm]{cth.pdf}
      \hspace{1em}
    \end{beamercolorbox}%
  }
  \vskip0pt%
}
\setbeamertemplate{navigation symbols}{}

\begin{document}
{
\setbeamertemplate{logo}{\includegraphics[height=2cm]{sysbio.jpg}\vspace{200pt}}
\begin{frame}
\maketitle
\end{frame}
}

\begin{frame}{Today's goals}
  \begin{itemize}
  \item A working Git setup
  \item Enough knowledge to start tracking your work with Git
  \end{itemize}
\end{frame}

\begin{frame}{Today's goals}
  Things we won't cover today (but that will be part of the next workshop):

  \begin{itemize}
    \item Incorporating changes from others (a.k.a. merging)
    \item Handling conflicts when merging
    \item Collaborating with other people using Git
  \end{itemize}
\end{frame}

\begin{frame}{}
  \center
  \Huge Why version control?
\end{frame}

\begin{frame}{Why version control?}
  \center
  \Huge It lets us track versions.
\end{frame}

\begin{frame}[fragile]{Why version control?}
  \center
  \large \mintinline{text}{Copy of Thesis.new.FINAL.fixed_comments(2).docx}
\end{frame}

\begin{frame}{Why version control?}
  \center
  \Huge It lets us track versions.

  \huge (In a sane way.)
\end{frame}

\begin{frame}{Why version control?}
  \center
  \Huge It lets us easily document changes.
\end{frame}

\begin{frame}{Why version control?}
  \center
  \Huge It simplifies collaboration.
\end{frame}

\begin{frame}{}
  \center
  \Huge Why Git?
\end{frame}

\begin{frame}{Why Git?}
  \center
  \Huge There are many version control systems.
  \pause

  \huge (CVS, SVN, Mercurial, Darcs, ...)
\end{frame}

\begin{frame}{Why Git?}
  \center
  \Huge Git is the most popular one.
  \pause

  \large (Used by Google, Facebook, Microsoft, Twitter, Netflix, Android, Linux, ...)
\end{frame}

\begin{frame}{Why Git?}
  \center
  \Huge Lots of users means lots of available help.
\end{frame}

\begin{frame}{Why Git?}
  \center
  \Huge Best reason: I got to pick it.
\end{frame}

\begin{frame}{}
  \center
  \Huge Git terminology
\end{frame}

\begin{frame}{Git terminology}
  \center
  \Huge Git tracks files and content you add in a database
\end{frame}

\begin{frame}{Git terminology}
  \center
  \Huge Everything resides in your work directory
\end{frame}

\begin{frame}{Git terminology}
  \center
  \Huge Everything resides in your work directory

  \huge This is called a ``repository''
\end{frame}

\begin{frame}{Git terminology}
  \center
  \Huge Each repository is independent from each other
\end{frame}

\begin{frame}{Git terminology}
  \center
  \Huge Keep things that are unrelated in different repositories
\end{frame}

\begin{frame}{Git terminology}
  \center
  \Huge Changes stored in Git are called ``commits''
\end{frame}

\begin{frame}{Git terminology}
  \center
  \Huge The commit history is a graph
\end{frame}

\begin{frame}{Git terminology}
  \center
  \Huge The commit history is a graph

  \Large $ \circ \rightarrow \circ \rightarrow \bullet $
\end{frame}

\begin{frame}{Git terminology}
  \center
  \Huge Git allows you to create ``branches'' inside your repository
\end{frame}

\begin{frame}{Git terminology}
  \center
  \Huge Branches lets you work on different versions without overwriting your changes
\end{frame}

\begin{frame}{Git terminology}
  \center
  \Huge The main branch is called ``master''
\end{frame}

\begin{frame}{Git terminology}
  \center
  \Huge Branches are easy to create and easy to use
\end{frame}

\begin{frame}{Git terminology}
  \center
  \Huge Git keeps a pointer called ``HEAD''

  \Large This points to your current place in the graph
\end{frame}

\begin{frame}{Git terminology}
  \center
  \Huge HEAD changes when you move between branches or in the commit history
\end{frame}

\begin{frame}{Git terminology}
  \center
  \Huge HEAD changes when you move between branches or in the commit history

  \huge and when you add new commits
\end{frame}

\begin{frame}{}
  \center
  \Huge Exercise 0: Basic configuration
\end{frame}

\begin{frame}[fragile]{Exercise 0: Basic configuration}
  Outcome:

  \begin{minted}{text}
    $ git config --get user.name
    Matthias Nilsson
    $ git config --get user.email
    mattiasn@chalmers.se
  \end{minted}
\end{frame}

\begin{frame}[fragile]{Exercise 0: Basic configuration}
  Commands:

  \begin{minted}{text}
    $ git config --global user.name "Your Name"
    $ git config --global user.email "you@chalmers.se"
  \end{minted}
\end{frame}

\begin{frame}{}
  \center
  \Huge Exercise 1: Create a new repository
\end{frame}

\begin{frame}[fragile]{Exercise 1: Create a new repository}
  Outcome:

  \begin{minted}{text}
    $ ls -a
    . .. .git
  \end{minted}
\end{frame}

\begin{frame}[fragile]{Exercise 1: Create a new repository}
  Commands:

  \begin{minted}{text}
    $ mkdir repo && cd repo
    $ git init
  \end{minted}
\end{frame}

\begin{frame}{}
  \center
  \Huge Exercise 2: Add your first file
\end{frame}

\begin{frame}[fragile]{Exercise 2: Add your first file}
  Outcome:

  \begin{minted}{text}
    $ git log
    commit e443e98a88b2f7175b213ca54b07668c69a6af18
    Author: Matthias Nilsson <mattiasn@chalmers.se>
    Date:   Wed May 6 13:02:12 2015 +0200

        initial commit
  \end{minted}
\end{frame}

\begin{frame}[fragile]{Exercise 2: Add your first file}
  Commands:

  \begin{minted}{text}
    $ touch file1
    $ git add file1
    $ git commit
  \end{minted}
\end{frame}

\begin{frame}{}
  \center
  \Huge Anatomy of a commit
\end{frame}

\begin{frame}[fragile]{Anatomy of a commit}
  \begin{minted}{text}
    commit e443e98a88b2f7175b213ca54b07668c69a6af18
    Author: Matthias Nilsson <mattiasn@chalmers.se>
    Date:   Wed May 6 13:02:12 2015 +0200

        initial commit
  \end{minted}
\end{frame}

\begin{frame}{}
  \center
  \Huge Writing commit messages
\end{frame}

\begin{frame}{Writing commit messages}
  Good commit messages:
  \begin{enumerate}
    \item Start with a short description (in imperative form) of what was done
    \item Contains detailed explanatory text (if necessary) after a blank line
    \item Explain WHAT was done and WHY
  \end{enumerate}
\end{frame}

\begin{frame}[fragile]{Writing commit messages}
  \begin{minted}{text}
    Add support for navigation by touch

    Previous version would, if in darkness, run or walk
    straight into things, which would cause embarrassment,
    injury, and/or death.

    This change allows subject to navigate safely in
    the dark.

    Closes #45512, #55237
  \end{minted}
\end{frame}

\begin{frame}{}
  \center
  \Huge Getting other people's code
\end{frame}

\begin{frame}{Getting other people's code}
  \center
  \Huge Email
\end{frame}

\begin{frame}{Getting other people's code}
  \center
  \Huge Download
\end{frame}

\begin{frame}{Getting other people's code}
  \center
  \Huge Cloning with Git
\end{frame}

\begin{frame}{Getting other people's code}
  \center
  \Huge Forking on GitHub

  \Large (And then cloning with Git)
\end{frame}

\begin{frame}{}
  \center
  \Huge Exercise 3: Clone a repository
\end{frame}

\begin{frame}[fragile]{Exercise 3: Clone a repository}
  Outcome:

  \begin{minted}{text}
    $ ls git_intro
    README.md
  \end{minted}
\end{frame}

\begin{frame}[fragile]{Exercise 3: Clone a repository}
  Commands:

  \begin{minted}{text}
    $ git clone https://github.com/SysBioChalmers/git_intro.git
  \end{minted}
\end{frame}

\begin{frame}{}
  \center
  \Huge Exercise 4: Fork a repository
\end{frame}

\begin{frame}{Exercise 4: Fork a repository}
  \center
  \large \mintinline{text}{https://github.com/SysBioChalmers/git_intro}
  \pause

  \Large Click on "Fork"
\end{frame}

\begin{frame}{Exercise 4: Fork a repository}
  \center
  \Huge Now clone it!
  \pause

  \LARGE (Make sure you use the SSH clone URL!)
\end{frame}

\begin{frame}{}
  \center
  \Huge Exercise 5: Push changes
\end{frame}

\begin{frame}{Exercise 5: Push changes}
  Outcome:

  Check repo on GitHub
\end{frame}

\begin{frame}[fragile]{Exercise 5: Push changes}
  Commands:

  \begin{minted}{text}
    $ git diff
    $ git add
    $ git commit
    $ git push
  \end{minted}
\end{frame}

\begin{frame}{}
  \center
  \Huge Exercise 6: Branches
\end{frame}

\begin{frame}[fragile]{Exercise 6: Branches}
  Outcome:

  \begin{minted}{text}
    $ git branch
      bar
    * foo
      master
  \end{minted}
\end{frame}

\begin{frame}[fragile]{Exercise 6: Branches}
  Commands:

  \begin{minted}{text}
    $ git branch bar
    $ git branch
    $ git checkout bar
    $ git checkout -b foo
  \end{minted}
\end{frame}

\begin{frame}[fragile]{Exercise 6: Branches}
  My regular work flow:

  \begin{minted}{text}
    $ git checkout -b my-branch
    $ git push -u
  \end{minted}
\end{frame}

\begin{frame}{Exercise 6: Branches}
  \center
  \Large
  Last command creates the branch on GitHub and tells Git to track changes to it
  \pause

  This is something you almost always want to do

\end{frame}

\begin{frame}{}
  \center
  \Huge A word on files
\end{frame}

\begin{frame}{A word on files}
  \center
  \huge There are text files and binary files
  \pause

  \Large Text files are human readable
  \pause

  \Large Binary files are images, PDFs, etc.
\end{frame}

\begin{frame}{A word on files}
  \center
  \Huge Git can track both text and binary files
  \pause

  \Large But we can't see what was changed in a binary file,

  \Large so we miss out on a lot of the things that Git is great at.
\end{frame}

\begin{frame}{A word on files}
  \center
  \Huge This is something you need to take into consideration
  \pause

  \Large (Personally, I avoid binary files, because I find text files
  easier to work with.)
\end{frame}

\begin{frame}{}
  \center
  \Huge What to do in case of fire
\end{frame}

\begin{frame}{What to do in case of fire}
  \begin{enumerate}
    \item Keep calm, most things can be fixed
    \item Read the documentation (can be confusing)
    \item Pro Git book: https://git-scm.herokuapp.com/book/en/v2
    \item Google it (StackOverflow has lots of help)
  \end{enumerate}
\end{frame}

\begin{frame}{}
  \center
  \Huge Questions?
\end{frame}

\end{document}